\documentclass{article}
\usepackage[svgnames]{xcolor}
\usepackage[a0paper,landscape,margin=3cm]{geometry}
\usepackage{anyfontsize}
\usepackage{tikz}
\usepackage{mathpazo}
\usepackage{multicol}
\usepackage{tikz}
\usepackage{graphicx}
\usetikzlibrary{matrix,fadings,arrows,trees,calc,positioning,decorations,automata,fit,backgrounds}
\usepackage{blindtext}
\usepackage[inkscapeformat=png]{svg}
\pagecolor{red!5}

\columnsep=50pt
\columnseprule=2pt

\renewcommand{\section}[1]{
    \begin{center}
        \begin{tikzpicture}
            \draw node[fill=white!10, text width=0.9\linewidth, text centered, inner sep=30pt, rounded corners=5pt, draw=red!80]
            {\textbf{#1}};
        \end{tikzpicture}
        
    \end{center}

}

\begin{document}


    \fontsize{55}{65}
    \selectfont
     \begin{tikzpicture}
            
        \node[inner sep=0pt] (logo) {\includesvg{SRH_Bildung.svg}}; 
        \draw node[text width=0.88\linewidth,text centered,rounded corners=5pt,inner sep=30pt,right=100pt of logo] (title){
            {Application of Natural Language Processing in an E-commerce industry}\\[0.5cm]
            \fontsize{50}{55}
            \selectfont
            {Predicting multilevel product category} \\[1cm]  
            \fontsize{40}{50}
            \selectfont
            Shoney Arickathil, Prof. Dr. Gerd Moeckel \\[0.5cm]
            \fontsize{50}{50}
            \selectfont
            Applied Computer Science, SRH Hochschule, Heidelberg
            };
    \begin{scope}[on background layer]
        \node [fill=Red!10, fit={(logo) (title)}] {};
    \end{scope}

    \end{tikzpicture}


\vspace{1cm}
\begin{multicols}{2}

   
    \fontsize{30}{40}
    \selectfont
    \section{Introduction}

    Ever increasing demand of online platforms trending online shopping huge number of products for online sales  need to add category for the products Un matching product taxonomy from various source. Existing taxonomy unsupervised learning model need of the hour to create an AI model to predict the category tree in which the product belongs to based on the product features such as description, manufacture.
    
    \blindtext
    \section{Problem statement and objective}
    Suppliers and manufactures provide the product details which must be analyzed and fit into the existing vocabular of the product category taxonomy. Make deployment ready products taxonomy. If a supplier name a certain product category as "Engine oil" and the category already existing is "Motor Oil". In such case importing the category from the supplier may lead to having two categories at same level. Imagine this issue with thousands of products solds on the ecommerse website. The ecomerse webiste will need to manage many categories. This paper researched on implementation of a text based classification model for predicting the category of the product based on product features. 
    \blindtext
    
\end{multicols}
\begin{minipage}{.70\textwidth}
   
    \centering
    \fontsize{30}{40}
    \selectfont
    \section{Research Questions}
    njnokliunoinoinoioinoin
   
\end{minipage}
\begin{minipage}{0.20\textwidth}
    \centering
    \fontsize{30}{40}
    \selectfont
    \section{Research Questions}
    
   
\end{minipage}

\begin{center}
    \section{References}
    
\end{center}
\end{document}