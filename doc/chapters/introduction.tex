\chapter{Introduction}

\section{Topic overview}

While working at an E-commerce company selling automobile parts, author realized that various manufacture and suppliers of similar kind of automobile parts may have different product taxonomy. Defining the product taxonomy is a prior task of product deployment. Manually defining the product taxonomy requires a significant amount of time and are prone to error. \\

E-commerce industry is growing, and many products are sold online. Use of machine learning model for suggesting the products to the customer, auto completing the search text, auto filtering of products based on the text provided by the user describing the product are few of the tasks which is managed by artificial intelligence. All the tasks can only be performed appropriately only if the product taxonomy is well-defined. \\

 

\section{Problem statement and Research question}

\begin{itemize}
    \item RQ1: How to define product features as an input for machine learning model?
    \item RQ2 : Which type of neural network is the best suitable for text classification in terms of product classification?
    \item RQ3 :  How should the model parameters be defined to optimize the neural network?
\end{itemize}

\section{Methodology}

Machine learning model to text classification for predicting the product taxonomy is a methodology in which a lot of research have been conducted \parencite{AliCevahir.} \parencite{Gupta.20062016}. In this thesis, author classifies the product by representing the features of the product into vector of entire vocabulary of dataset. This approach is inspired by \parencite{sean}, in which the language of the name is predicted based on the pattern of characters in name input. 

\section{Thesis structure}

In the first chapter, gives the introduction of the thesis along with research questions and methodology.

In the second chapter, defining or selecting the features from a document is described. It introduces various analytic engine tools and methods for feature selection. 

The third chapter describes various methods to standardize the text data and preprocess it before making it as an input to the classification model.

The fourth chapter focuses on retaining the missing textual data from the dataset from which the classification model will be predicting the product category. 

The fifth chapter is a guideline on vectoring or transforming the text into machine usable numerical form called feature extraction.

In the sixth chapter, how author classifies the product by representing the features of the product into vector of entire vocabulary of dataset is depicted. The architecture of neural network used for the prediction is described in this chapter. Mathematical equations on probability distribution, Softmax and why to apply log to softmax is formulated. 

In the seventh chapter, the model is trained to classification. The experimentation and analysis of model with different approaches and varying model parameters is conducted.

The eight chapter, is research on creating a knowledge graph with natural language processing.