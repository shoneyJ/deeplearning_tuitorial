\chapter{Introduction}

\section{Background and context}

While working at an e-commerce industry, specializing in sales of automobile parts, author realized that manufactures and suppliers of similar automobile parts often utilize different product taxonomies. Product taxonomy is hierarchical multi level categorization of the product.  Suppliers rely on online platforms to reach out to the customers. In the business context, the term "supplier" encompasses entity contributing to production and distribution of goods and services. Therefore, in this thesis, the term "supplier" include both manufactures or any source which provides information on the product for sale. Suppliers are business partners who provide the stocks of products along with associated product details.

However, due to inconsistencies in the product taxonomies among various suppliers, directly importing these product taxonomies into the online platform is not feasible. The inconsistency could be due to use of different vocabulary for the similar product. For example, supplier A may use the term "Motor oil" and supplier B may use the term "Engine oil" for the same product. The process of product deployment is as following: suppliers furnish the product details, online retailers update the product taxonomies with their specific vocabulary. Once the taxonomies are updated these products details are accessible for the customers.

Defining the product taxonomy is a prior and crucial task of product deployment. A well-defined product taxonomy enables customers to search the product with the fewest possible clicks.  Manually defining the product taxonomies requires a significant amount of time and are prone to human error. 

The e-commerce industry is growing substantially, and wide range of products being offered online. Utilization of machine learning model for recommending the products to the customer, auto completing the search text, automatically filtering of products based on the product description provided by the customer are few examples of function facilitated by artificial intelligence. It is important to note that successful execution of all these functionalities primary requirement is to have a well-defined product taxonomy.

 Customer review monitoring system uses natural language processing to understand the semantic meaning of the comment. Determining whether a customer is satisfied or dissatisfied based on their review can be a complex task. By understanding the semantic meaning helps to identify the negatively commented review regarding a product or service. The reviews are classified as either positive or negative, similar to the classification problem of predicting the product category. 
 
 In this paper, author equates on mathematical functionality of probability distribution and learning mechanism of \acl*{RNN}. These insights are applicable to wide range of classification problem.

 Apart from the classification problem, this paper presents the concept of knowledge base. Online retailers strive to engage with their customers effectively through tools like chatbot, capable of  answering frequently asked question. The queries asked by the customers and their relevant answers are in the form of an unstructured textual data. In this thesis, author presents a prototype for converting the unstructured textual data into a directed acyclic graph. These graphs, in conjunction with the question generator machine learning model,  can function as a knowledge base. Knowledge base serve as a foundation for chatbot applications to traverse through the graph structure, locate relevant leafs nodes, and formulate human-readable response.

\section{Research questions}

\begin{enumerate}[label=\textbf{RQ\arabic*:}]
    \item What are the use cases in which natural language processing and machine learning model can be utilized in an E-Commerce industry?
    \item What are the use cases in E-commerce industry akin to classification problem?
    \item For product categorization, how to define product features as an input for machine learning model?
    \item How does a machine learn patterns for classification?
    \begin{enumerate}[label=\textbf{SRQ\arabic*:}]
        \item What are the mathematical equations behind the machine learning process?
        \item What are the mathematical reasons for applying a certain Pytorch function during the training process? 
    \end{enumerate}

    \item What algorithm could be used to train the machine learning model to predict the product taxonomy?
    \item Which are the tasks indirectly dependent on a well-defined product taxonomies? 
    \item What are the future research areas around the created product category prediction model?
\end{enumerate}

\section{Methodology}

\subsection{Start with small prototype}

Developing a machine learning model to predict the multi level product categories has series of task and is a complex process. In this paper, to reduce the complexity and primarily focus on the classification problem, author choose to develop a model which predicts the lowest level of category only based on one feature that is the name of the product. Once the smaller working prototype is developed, more features can be integrated along with multiple levels of category. 

% \subsection{Data collection}
% The labelled training data set belongs to an E-commerce industry specializing in automotive parts \footnote{\url{www.retromotion.com}; Accessed on : 05-09-2023}. In this project, supervised classification model is developed using  the pair of feature set and label. These data from various suppliers will be processed and stored into search analytic engine-Elastic search. 

\subsection{Ideate: Vocabulary pattern}
Initial approach of finding a solution to product categorization is to analyze already existing solution to a different type of classification problem. For example, \parencite{sean} demonstrates the working of character level RNN. In which, the pattern of series of characters forming a name of a person are classified to predict which language the name belongs to. Analyzing this example author created a vocabulary level RNN to predict the lowest level of category based on the pattern of product name.

Machine learning model to text classification for predicting the product taxonomy is a methodology in which a lot of research have been conducted. \cite{AliCevahir.} implements classification model at every level using the \acl*{KNN} classifier. Like wise \parencite{Gupta.20062016}  proposes a distributional semantics representation for predicting the product taxonomies.  Some classifying methods are supervised classification, decision trees, naive Bayes classifier and max entropy classifiers \parencite{BirdKleinLoper09}

\subsection{Classification model evaluation}

In this thesis, author has recognized the complexity of task and has opted to base prediction solely on the product feature pattern. Analyzing the confusion matrix of predicted vs actual category will provide a visual evaluation of the model. The result of the model for all the existing data of product will be stored in search analytic engine Elastic search in order to manually check the falsely predicted category.

\section{Structure of the Thesis}

In the first chapter, gives the introduction of the thesis along with research questions and methodology.

In the second chapter, defining or selecting the features from a document is described. It introduces various analytic engine tools and methods for feature selection. 

The third chapter describes various methods to standardize the text data and preprocess it before making it as an input to the classification model.

The fourth chapter focuses on retaining the missing textual data from the dataset from which the classification model will be predicting the product category. 

The fifth chapter is a guideline on vectoring or transforming the text into machine usable numerical form called feature extraction.

In the sixth chapter, how author classifies the product by representing the features of the product into vector of entire vocabulary of dataset is depicted. The architecture of neural network used for the prediction is described in this chapter. Mathematical equations on probability distribution, Softmax and why to apply log to softmax is formulated. 

In the seventh chapter, the model is trained to classification. The experimentation and analysis of model with different approaches and varying model parameters is conducted.

The eight chapter, is research on creating a knowledge graph with natural language processing.