\chapter{Development environment and hardware setup}

\section {Ubuntu server 20.04 configuration}

\section {Python virtual environment} \label {pyenv}

pyenv facilitaes to isolate python version dependent source codes and avoid version conflicts. A virtual environment will install packages to a specific python module instad of installing globally. Asuming the directrory root directory name is "dev", follow linux bash code to activate virtual environment. 
The bash script below will create and activate virtual environment and all the packages will be installed within the venv directory
\begin{lstlisting}[language=bash]
    pyenv install 3.10 /
    cd ~/dev /
    pyenv local 3.10 /
    python -m venv .venv /
    source .venv/bin/activate

\end{lstlisting}

Compatible python clients to interact with project dependent module installation can be installed by specifying the version range.
Python Elastic search client's official low version installation details are mentioned on elasticsearch-py \footnote{https://elasticsearch-py.readthedocs.io/en/6.8.2/} website.

vGPU \footnote{https://docs.nvidia.com/grid/5.0/grid-vgpu-user-guide/index.html}

\begin{lstlisting}[language=Python]
    python -m pip install "elasticsearch>=6.4.0,<7.0.0"
\end{lstlisting}

\section {Docker containers - CouchDB, Elastic search, Redis}

\section {Hyper-v virtualization and networking}

\subsection{vGPU installation}

\subsection{Networking}

\subsection{SSH}


\section{Cloud computing}

